\documentclass[letterpaper]{article}
\usepackage[T1]{fontenc}
\usepackage[utf8]{inputenc}
\usepackage[spanish,es-tabla]{babel}
\usepackage[lmargin=3cm,rmargin=3cm,top=3cm,bottom=3cm]{geometry}
\usepackage{amsmath,amsfonts}
\usepackage{mathtools}
\usepackage{graphicx}
\usepackage{lipsum}
\usepackage{titling}
\usepackage{fancyhdr}
\usepackage{setspace}
\usepackage[natbibapa]{apacite}
\usepackage{caption}
\usepackage{float}
\usepackage{dirtytalk}

\lhead{Universidad del Bío-Bío\\ Facultad de ingeniería \\ Escuela de Ingeniería Industrial}
%\lhead{\begin{picture}(0,0) \put(0,0){\includegraphics[height=10mm]{logos/escuela}} \end{picture}}
\rhead{\begin{picture}(0,0) \put(-50,0){\includegraphics[height=10mm]{logos/escudo_ubb_2}} \end{picture}}


%\setlength{\parindent}{0pt}
\bibliographystyle{apacite}
\pagestyle{fancy}


\usepackage{titlesec}
\titleformat*{\section}{\large\bfseries\centering}
\titleformat*{\subsection}{\normalsize\bfseries}

\begin{document}
\vspace*{0.5\baselineskip}
\begin{center}
\begin{Large}
\textbf{\underline{Trabajo N\textsuperscript{\underline{o}}2}}
\end{Large}\\
\vspace*{0.5\baselineskip}
\textbf{Optimización Lineal} \\
\vspace*{0.5\baselineskip}
\begin{footnotesize}
\textbf{Código}: 430373\\
\textbf{Semestre}: 2019-2
\end{footnotesize}
\end{center}

\noindent \textbf{Profesor}: Dr Carlos Obreque Niñez  \hfill Ayudante: Alex Barrales Araneda\\
\noindent \textbf{Fecha}: \today

\subsection*{Problema 1}
Considere el siguiente modelo de PL:
\begin{align*}
\mbox{Minimizar }&Z = x_1 + x_2 \\
s.a.\\
&-4x_1 + 9x_2 \leq 36\\
&2x_1 + 3x_2 \leq 18\\
&-x_1 + 4x_2  \geq 4\\
&x_1 \: s.r.s.;x_2 \geq 0\\
\end{align*}

\begin{itemize}
\item Grafique la región factible
\item Encuentre la solución óptima e indique gráficamente a cuál vértice corresponde.
\item Obtenga la solución básica factible correspondiente a la solución óptima.
\item Determine dos soluciones básicas factibles e indique gráficamente a qué puntos extremos corresponden.
\item Proponga una nueva función objetivo de tal manera que el problema tenga múltiples soluciones óptimas. 
\end{itemize}

\subsection*{Problema 2}
Considere el siguiente modelo de PL:
\begin{align*}
\mbox{Minimizar }&Z = 10x_1 - 9x_2 \\
s.a.\\
&2x_1 + x_2 \geq 0\\
&-2x_1 + x_2 \leq 20\\
&-5 \leq x_1 \leq 5\\
&0 \leq x_2 \leq 10\\
\end{align*}

\begin{itemize}
\item Encuentre la solución óptima utilizando el método gráfico.
\item Obtenga la solución básica factible correspondiente a la solución óptima.
\item Determine dos soluciones básicas factibles e indique gráficamente a qué punto extremo corresponde.
\end{itemize}

\end{document}

