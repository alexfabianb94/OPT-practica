\documentclass[letterpaper]{article}
\usepackage[T1]{fontenc}
\usepackage[utf8]{inputenc}
\usepackage[spanish,es-tabla]{babel}
\usepackage[lmargin=3cm,rmargin=3cm,top=3cm,bottom=3cm]{geometry}
\usepackage{amsmath,amsfonts}
\usepackage{mathtools}
\usepackage{graphicx}
\usepackage{lipsum}
\usepackage{titling}
\usepackage{fancyhdr}
\usepackage{setspace}
\usepackage[natbibapa]{apacite}
\usepackage{caption}
\usepackage{float}
\usepackage{dirtytalk}

\lhead{Universidad del Bío-Bío\\ Facultad de ingeniería \\ Escuela de Ingeniería Industrial}
%\lhead{\begin{picture}(0,0) \put(0,0){\includegraphics[height=10mm]{logos/escuela}} \end{picture}}
\rhead{\begin{picture}(0,0) \put(-50,0){\includegraphics[height=10mm]{logos/escudo_ubb_2}} \end{picture}}


%\setlength{\parindent}{0pt}
\bibliographystyle{apacite}
\pagestyle{fancy}


\usepackage{titlesec}
\titleformat*{\section}{\large\bfseries\centering}
\titleformat*{\subsection}{\normalsize\bfseries}

\begin{document}
\vspace*{0.5\baselineskip}
\begin{center}
\begin{Large}
\textbf{\underline{Guía N\textsuperscript{\underline{o}}4}}
\end{Large}\\
\vspace*{0.5\baselineskip}
\textbf{Optimización Lineal} \\
\vspace*{0.5\baselineskip}
\begin{footnotesize}
\textbf{Código}: 430373\\
\textbf{Semestre}: 2019-2
\end{footnotesize}
\end{center}

\noindent \textbf{Profesor}: Dr Carlos Obreque Niñez  \hfill Ayudante: Alex Barrales Araneda\\
\noindent \textbf{Fecha}: \today

\subsection*{Problema 1}
Una persona recibe un premio de 10 millones de pesos en una lotería y le aconsejan que los invierta en dos tipos de acciones, A y B. Las acciones tipo A tienen más riesgo pero producen un beneficio anual del 10\%. Las acciones tipo B son más seguras, pero producen sólo el 7\% anual. Después de varias deliberaciones decide invertir como máximo 6 millones de pesos en la compra de acciones tipo A y por lo menos 2 millones de pesos en la compra de acciones tipo B. Además, decide que lo invertido en A sea, por lo menos, igual a lo invertido en B. 

Defina variables de decisión apropiadas y escriba un modelo de programación lineal para resolver este problema.

\subsection*{Problema 2}
En un salón de banquetes se tienen programados banquetes durante los siguientes cinco días: Los requisitos de manteles por banquete son:

\begin{table}[H]
\centering
\begin{tabular}{lrrrrr}
\hline
\textbf{Banquete}           & 1  & 2  & 3   & 4   & 5   \\ \hline
\textbf{Número de Manteles} & 80 & 60 & 100 & 130 & 200 \\ \hline
\end{tabular}
\end{table}

El problema del administrador es que se requieren manteles diferentes a los que actualmente se usan, por lo que tendrá que comprar manteles nuevos. El costo de cada mantel es de \$40 y el costo de mandarlo a la lavandería bajo servicio urgente para tenerlo listo a los dos días después es de \$10 por cada mantel. ¿Cuál es el modelo que le permitirá al administrador cumplir con los
requisitos y además minimizar el costo total?

\subsection*{Problema 3}
Una empresa productora de jugos de fruta, cada semana, utiliza una sola máquina durante 150 horas para destilar jugo de naranja y de limón en concentrados que se almacenan en dos tanques separados de 1500 litros cada uno antes de congelarlos. La máquina puede
procesar 25 litros de jugo de naranja por hora, pero sólo 20 litros de jugo de limón por hora. Cada litro de jugo de naranja cuesta \$1.50 y pierde 30\% de contenido de agua al destilarse en concentrado. El concentrado de jugo de naranja se vende después en \$6 por litro. Cada litro de jugo de limón cuesta \$2.00 y pierde 25\% de contenido de agua al destilarse en concentrado. El concentrado de jugo de limón se vende después en \$8 por litro. Defina las variables de decisión y formule un modelo de programación lineal para determinar un plan de producción que maximice la ganancia para la siguiente semana.

\subsection*{Problema 4}
Un pequeño taller arma dispositivos mecánicos, ya sea como un producto terminado que entrega al mercado, o como un proceso intermedio para entregar a una gran fábrica. Trabajan 3 personas en jornadas de 40 horas semanales. Dos de estos obreros no calificados reciben \$0.4 por hora, y el tercero, un obrero calificado, recibe \$0.6 por hora. Los tres están dispuestos a trabajar hasta 10 horas adicionales a la semana con un salario 50\% superior durante este período. Los costos fijos semanales son de \$800. Los gastos de operación variables son de \$1.0 por hora de trabajo de obrero no calificado y \$2.4 por hora de obrero calificado. Los dispositivos mecánicos sin terminar son vendidos a la fábrica a \$6.5 cada uno. El taller tiene un contrato bajo el cual debe entregar 100 de estos dispositivos semanalmente a la fábrica. El dueño del taller tiene como política el producir no más de 50 dispositivos sin terminar a la semana por sobre el contrato que tiene comprometido.

Los dispositivos terminados se venden a \$15 cada uno sin restricciones de mercado. Se requieren 0.5 horas de obrero no calificado y 0.25 horas de obrero calificado para producir un dispositivo sin acabar listo para entregar a la fábrica. Uno de estos dispositivos sin terminar puede ensamblarse y dejarlo terminado agregándole 0.5 horas de trabajador calificado. Un dispositivo terminado listo para entregar al mercado se puede producir con 0.6 horas de obrero no calificado y 0.5 horas de obrero calificado.

El administrador del taller desea saber la cantidad de dispositivos mecánicos sin terminar, los terminados y los que se terminan a partir de otros no terminados. También, las horas de tiempo normal y extraordinarias que trabajan los obreros calificados y los no calificados. Defina las variables de decisión y formule un modelo de programación lineal que permita programar la producción de modo de maximizar las utilidades.

\end{document}

