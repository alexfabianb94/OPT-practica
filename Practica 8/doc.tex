\documentclass[letterpaper]{article}
%\documentclass[legalpaper]{article}
\usepackage[T1]{fontenc}
\usepackage[utf8]{inputenc}
\usepackage[spanish,es-tabla]{babel}
\usepackage[lmargin=3cm,rmargin=3cm,top=3cm,bottom=3cm]{geometry}
\usepackage{amsmath,amsfonts}
\usepackage{mathtools}
\usepackage{graphicx}
\usepackage{lipsum}
\usepackage{titling}
\usepackage{fancyhdr}
\usepackage{setspace}
\usepackage[natbibapa]{apacite}
\usepackage{caption}
\usepackage{float}
\usepackage{dirtytalk}

\lhead{Universidad del Bío-Bío\\ Facultad de ingeniería \\ Escuela de Ingeniería Industrial}
%\lhead{\begin{picture}(0,0) \put(0,0){\includegraphics[height=10mm]{logos/escuela}} \end{picture}}
\rhead{\begin{picture}(0,0) \put(-50,0){\includegraphics[height=10mm]{logos/escudo_ubb_2}} \end{picture}}


%\setlength{\parindent}{0pt}
\bibliographystyle{apacite}
\pagestyle{fancy}


\usepackage{titlesec}
\titleformat*{\section}{\large\bfseries\centering}
\titleformat*{\subsection}{\normalsize\bfseries}


\usepackage{filecontents}
\begin{filecontents}{\jobname.bib}
@book{Taha:1998,
  title={Investigaci{\'o}n de operaciones},
  author={Taha, H.A.},
  isbn={9786073207966},
  year={2012},
  publisher={Prentice Hall}
}
\end{filecontents}

\usepackage{natbib}










\begin{document}
\vspace*{0.5\baselineskip}
\begin{center}
\begin{Large}
\textbf{\underline{Práctica N\textsuperscript{\underline{o}}8}}
\end{Large}\\
\vspace*{0.5\baselineskip}
\textbf{Optimización Lineal} \\
\vspace*{0.5\baselineskip}
\begin{footnotesize}
\textbf{Código}: 430373\\
\textbf{Semestre}: 2019-2
\end{footnotesize}
\end{center}

\noindent \textbf{Profesor}: Dr Carlos Obreque Niñez  \hfill Ayudante: Alex Barrales Araneda\\
\noindent \textbf{Fecha}: 3 de diciembre de 2019

\subsection*{Problema 1}
Considere el siguiente modelo de PL:
\begin{align*}
\mbox{Maximizar }&Z = x_1 + x_2\\
s.a.\\
&2x_1 + x_2 \geq 2\\
&-x_1 + 4x_2 \geq 2\\
&5x_1 + 7x_2 \leq 35\\
&x_1 \geq 0 ; x_2 \geq 0\\
\end{align*}

\begin{itemize}
\item Escriba el problema dual y obtenga su solución óptima.
\end{itemize}
\subsection*{Problema 2}
Considere el siguiente modelo de programación lineal:
\begin{align*}
\mbox{Maximizar }&Z = 2x_1 + x_2 \\
s.a.\\
&3x_1 + 5x_2 \geq 15\\
&x_1 + x_2 \leq 15\\
&3x_1 + x_2  = 15\\
&x_1 \geq 0 ;x_2 \geq 0\\
\end{align*}

\begin{itemize}
\item Formule el dual del problema.
\item Determine la solución óptima mediante el método Simplex Dual.
\end{itemize}

\newpage
\begin{table}[H]
\centering
%\begin{footnotesize}
\begin{tabular}{lllll}
\hline
            & Mesa 4s & Mesa 6s & Silla & Mueble \\
\hline
Corte       & 1       & 1.5     & 0.4   & 2      \\
Ensamble    & 1.2     & 1.4     & 0.1   & 3      \\
Terminación & 1.8     & 2       & 0.5   & 4      \\
\hline
Utilidad    & 3000    & 4300    & --    & 9000   \\
\hline
\end{tabular}
%\end{footnotesize}
\label{tabla:1}
\end{table}


\bibliographystyle{plainnat}
\bibliography{\jobname} 
\end{document}


