\documentclass[letterpaper]{article}
\usepackage[T1]{fontenc}
\usepackage[utf8]{inputenc}
\usepackage[spanish,es-tabla]{babel}
\usepackage[lmargin=3cm,rmargin=3cm,top=3cm,bottom=3cm]{geometry}
\usepackage{amsmath,amsfonts}
\usepackage{mathtools}
\usepackage{graphicx}
\usepackage{lipsum}
\usepackage{titling}
\usepackage{fancyhdr}
\usepackage{setspace}
\usepackage[natbibapa]{apacite}
\usepackage{caption}
\usepackage{float}
\usepackage{dirtytalk}

\lhead{Universidad del Bío-Bío\\ Facultad de ingeniería \\ Escuela de Ingeniería Industrial}
%\lhead{\begin{picture}(0,0) \put(0,0){\includegraphics[height=10mm]{logos/escuela}} \end{picture}}
\rhead{\begin{picture}(0,0) \put(-50,0){\includegraphics[height=10mm]{logos/escudo_ubb_2}} \end{picture}}


%\setlength{\parindent}{0pt}
\bibliographystyle{apacite}
\pagestyle{fancy}


\usepackage{titlesec}
\titleformat*{\section}{\large\bfseries\centering}
\titleformat*{\subsection}{\normalsize\bfseries}

\begin{document}
\vspace*{0.5\baselineskip}
\begin{center}
\begin{Large}
\textbf{\underline{Trabajo N\textsuperscript{\underline{o}}1}}
\end{Large}\\
\vspace*{0.5\baselineskip}
\textbf{Optimización Lineal} \\
\vspace*{0.5\baselineskip}
\begin{footnotesize}
\textbf{Código}: 430373\\
\textbf{Semestre}: 2019-2
\end{footnotesize}
\end{center}

\noindent \textbf{Profesor}: Dr Carlos Obreque Niñez  \hfill Ayudante: Alex Barrales Araneda\\
\noindent \textbf{Fecha}: \today

\subsection*{Problema 1}
Una cooperativa tiene una finca de 300 hectáreas que puede bombear un millón de metros cúbicos del acuífero adyacente. La cooperativa quiere usar la totalidad de la finca con fines agropecuarios y proyecta producir plátano y maíz y también sembrar pasto de pastoreo para la cría de ganado. Una hectárea de plátano requiere 10 mil metros cúbicos de agua y 40 horas de mano de obra. Una hectárea de maíz requiere cuatro mil metros cúbicos de agua y 12 horas de mano de obra. Una cabeza de ganado requiere media hectárea de pasto, 100 metros cúbicos de agua (incluyendo el agua para el pasto) y ocho horas de mano de obra. La cooperativa dispone de un capital de 100 millones de pesos y un total de ocho mil horas de mano de obra. Los costos de producción de una hectárea de plátano y de una de maíz son \$500.000 y \$100.000 respectivamente, mientras que la producción de ganado cuesta \$70.000 por cabeza. El ingreso bruto anual de la cooperativa es de \$1.100.000 por hectárea de plátano y de \$300.000 por hectárea de maíz. Una cabeza de ganado después de un año de engorde vale \$130.000.
\begin{itemize}
\item Formule un modelo de programación lineal para determinar el número máximo de cabezas de ganado; y las hectáreas de plátano y maíz que maximizan la ganancia neta de la cooperativa.
\end{itemize}

\subsection*{Problema 2}
El dueño de un fundo de la región desea determinar la cantidad de hectáreas de porotos y de tomates que tiene que sembrar esta temporada. Una hectárea sembrada con porotos produce 25 sacos de porotos y requiere 10 horas de trabajo. Una hectárea sembrada con tomates produce 10 cajones de tomates y requiere 4 horas de trabajo. El dueño puede vender toda la producción de porotos a 4 dólares el saco y toda la producción de tomates a 3 dólares el cajón. Dispone de 7 hectáreas de terreno y de 40 horas en la temporada para sembrar las semillas de ambos productos. Disposiciones gubernamentales especifican que el dueño del fundo debe tener una producción de porotos de por lo menos 30 sacos durante la temporada en curso. Por otra parte, es necesario asegurar que las hectáreas sembradas con porotos deben ser a lo menos el 60\% del total de hectáreas sembradas.

\begin{itemize}
\item Defina las variables de decisión apropiadas y formule un modelo de programación lineal para maximizar el ingreso total por la producción de porotos y tomates. 
\end{itemize}

\end{document}

